\section{Introduction}

For advice on how to write and structure a paper, see the following link (some of the content in this template is taken from this link):

\url{https://cs.stanford.edu/people/widom/paper-writing.html?s=03#intro}

\medskip

Read and bookmark the page above. You are recommended to follow the ``Stanford InfoLab's patented five-point structure for Introductions''. Unless there's a good argument against it, the Introduction should consist of five paragraphs answering the following five questions:

\begin{enumerate}
\item What is the problem?
\item Why is it interesting and important?
\item Why is it hard? (E.g., why do naive approaches fail?)
\item Why hasn't it been solved before? (Or, what's wrong with previous proposed solutions? How does mine differ?)
\item What are the key components of my approach and results? Also include any specific limitations.
\end{enumerate}

\paragraph{\textbf{A few notes on the use of \LaTeX.}} 
First, double quotes are done with \verb|``| and \verb|''|. For example, \verb|``double quotes''| yields ``double quotes''. Note that if you write \verb|"double quotes"|, you obtain "double quotes" (which doesn't look as nice).

Citations are done with the \verb@\cite@ command \cite{Goodfellow:2014}. 
Note that it is always preferable to be possible to read your text without the references. For example, sentences like the following should be avoided:

\begin{quote}
\sl
Generative adversarial networks were first introduced in \cite{Goodfellow:2014}.
\end{quote}
Instead, one can write:
\begin{quote}
\sl
Goodfellow et al. introduced and described Generative adversarial networks~\cite{Goodfellow:2014}.
\end{quote}

 
Bibliographic entries are placed in the file \textsf{Bibliography.bib}.
This is another citation \cite{McDonagh:2011dp} and this is another one \cite{Feller:2011ys}.
\textbf{Important:} in most cases, you do not need to create the bibliographic entry manually; instead, you can use a service like Google Scholar\footnote{Extracting bibtex entries from Google Scholar: \url{https://libguides.usask.ca/c.php?g=218034&p=1445751}}, which is recommended for you to find related work.

For advice on creating professional and clean tables with \LaTeX, see:

\url{https://texblog.org/2017/02/06/proper-tables-with-latex/}

Table~\ref{tab:example} is an example.

\subsection{Work Objectives}
\lipsum[2] % remove this

\subsection{Expected Contributions}
\lipsum[2] % remove this
