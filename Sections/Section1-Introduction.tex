\section{Introduction}

The Introduction is crucially important. By the time a referee has finished the Introduction, they've probably made an initial decision about whether to accept or reject the paper -- they'll read the rest of the paper looking for evidence to support their decision. A casual reader will continue on if the Introduction captivated them, and will set the paper aside otherwise. Again, the Introduction is crucially important.
Here is the Stanford InfoLab's patented five-point structure for Introductions. Unless there's a good argument against it, the Introduction should consist of five paragraphs answering the following five questions:

\begin{enumerate}
\item What is the problem?
\item Why is it interesting and important?
\item Why is it hard? (E.g., why do naive approaches fail?)
\item Why hasn't it been solved before? (Or, what's wrong with previous proposed solutions? How does mine differ?)
\item What are the key components of my approach and results? Also include any specific limitations.
\end{enumerate}

(Exercise: Answer these questions for the multiway sort example.)
Then have a final paragraph or subsection: "Summary of Contributions". It should list the major contributions in bullet form, mentioning in which sections they can be found. This material doubles as an outline of the rest of the paper, saving space and eliminating redundancy.
(Exercise: Write the bullet list for the multiway sort example.)
