\begin{table}[htb!]
\centering
\footnotesize
\begin{tabularx}{\textwidth}{@{}lXl@{}}
\toprule
\textbf{Activation Function} & \textbf{Formula} & \textbf{Description} \\
\midrule
Linear & \( a(x) = x \) & Suitable for regression tasks. \\
\addlinespace
Sigmoid & \( \sigma(x) = \frac{1}{1 + e^{-x}} \) & Maps inputs to (0, 1), for binary classification. \\
\addlinespace
Hyperbolic Tangent & \( \tanh(x) = \frac{e^{x} - e^{-x}}{e^{x} + e^{-x}} \) & Maps inputs to (-1, 1), offering richer gradients than sigmoid. \\
\addlinespace
ReLU & \( \text{ReLU}(x) = \max(0, x) \) & Encourages sparse activation, helps with gradient flow. \\
\addlinespace
Softmax & \( \text{Softmax}(z)_i = \frac{e^{z_i}}{\sum_{k=1}^K e^{z_k}} \) & Turns logits into probabilities, for multi-class classification. \\
\bottomrule
\end{tabularx}
\caption{Summary of the most common Activation Functions \cite{goodfellow_deep_2016}.}
\label{Tables:ActivationFunctions}
\end{table}
